\documentclass{beamer}
\usepackage{dsfont}
\usepackage{listings}
\usepackage{caption}
\usetheme{CambridgeUS}
\mode<presentation>
\title[Exploring R]{Exploring R via Fourier Series}
\institute[NIC]{Near Infinity Corporation}
\author[Ed MacDonald]{Ed MacDonald \\ \texttt{emacdona@nearinfinity.com}}
\renewcommand\mathfamilydefault{\rmdefault}
\usepackage{verbatim}
\definecolor{demph-color}{gray}{0.80}
\definecolor{listinggray}{gray}{0.90}

\setlength{\abovecaptionskip}{-2ex}
\setlength{\belowcaptionskip}{-2ex}

\setbeamertemplate{itemize items}[default]
\setbeamertemplate{enumerate items}[default]

\captionsetup{font=scriptsize,labelformat=empty}

\lstset{language=R} 
\lstset{backgroundcolor=\color{listinggray}}
\lstset{linewidth=90mm}
\lstset{frame=single}
\lstset{keywordstyle=\color{blue}}
\lstset{basicstyle=\tiny}
\lstdefinestyle{WithHighlights}{
   language={R}, 
   moredelim=**[is][\color{red}]{@}{@}
}

\begin{document}

\begin{frame}
   \titlepage
\end{frame}

\begin{frame}
   \frametitle{Fourier Series}
   \begin{block}{}
      Any periodic signal can be represented by an infinite sum of sinusoids.
   \end{block}
\end{frame}

\begin{frame}
   \frametitle{Fourier Series}
   \[
      \color<2->{demph-color}{x(t) = }
         \alert<2>{a_{0}}
         \color<2->{demph-color}{+ \sum_{n=1}^{\infty}}
         \alert<2>{a_{n}} 
         \color<2->{demph-color}{\cos n} 
         \alert<2>{\omega_{0}} 
         \color<2->{demph-color}{t + }
         \alert<2>{b_{n}} 
         \color<2->{demph-color}{\sin n} 
         \alert<2>{\omega_{0}} 
         \color<2->{demph-color}{t}
   \]
   \color<2->{demph-color}{where:}
   \begin{align}
      \alert<2>{\omega_{0}} 
      &\color<2->{demph-color}{= 2 \pi f_{0} = \frac{2\pi}{\alert<2>{T_{0}}}} \notag \\
      \alert<2>{a_{0}} 
      &\color<2->{demph-color}{= \frac{1}{T_{0}} \int_{T_{0}} x(t) \, dt} \notag \\
      \alert<2>{a_{n}} 
      &\color<2->{demph-color}{= \frac{2}{T_{0}} \int_{T_{0}} x(t) \cos n\omega_{0}t \, dt} \notag \\
      \alert<2>{b_{n}} 
      &\color<2->{demph-color}{= \frac{2}{T_{0}} \int_{T_{0}} x(t) \sin n\omega_{0}t \, dt} \notag  
   \end{align}
\end{frame}

\begin{frame}
   \frametitle{Fourier Series Approximation}
   \[
      \textcolor{red}{F_{N}}(t) = a_{0} + \sum_{n=1}^{\textcolor{red}{N}}
         a_{n} \cos n \omega_{0} t + 
         b_{n} \sin n \omega_{0} t
   \]
\end{frame}

\begin{frame}
   \frametitle{(Some) R Data Types}
   \begin{columns}
      \begin{column}{5cm}
         \begin{itemize}
            \item \alert<1>{Vectors}
            \item \alert<2>{Lists}
            \item \alert<3>{S3 Objects}
         \end{itemize}
      \end{column}
      \begin{column}{5cm}
         \only<1>{
            \begin{block}{}
               \begin{itemize}
                  \item Fundamental R data type
                  \item Scalars are represented as single element Vectors
                  \item Constructed with: c()
               \end{itemize}
            \end{block}
         }
         \only<2>{
            \begin{block}{}
               Lists are associative arrays. Like Hashes in Perl or Dictionaries in Python.
            \end{block}
         }
         \only<3>{
            \begin{block}{}
               S3 Objects are the old fashioned, type-unsafe way of doing OO in R. They are built on Lists much like Perl objects are built on Hashes.
            \end{block}
         }
      \end{column}
   \end{columns}
\end{frame}

\begin{frame}[fragile]
   \frametitle{So let's model a Fourier Series Approximation in R...}
   \[
      F_{N}(t) = \textcolor{red}{a_{0}} + \sum_{n=1}^{N}
      \textcolor{red}{a_{n}} \cos n \textcolor{red}{\omega_{0}} t + 
      \textcolor{red}{b_{n}} \sin n \textcolor{red}{\omega_{0}} t
   \]
   \begin{center}
   \begin{minipage}{100mm}
   \begin{lstlisting}[style=WithHighlights]
makeFSApprox <- function(T0, an, bn, trueFn){
   fsApprox <- list(
      @T0@ = T0,
      @an@ = an,
      @bn@ = bn,
      trueFn = trueFn
   );
   class(fsApprox) <- c("FSApprox");
   return(fsApprox);
}
   \end{lstlisting}
   \end{minipage}
   \end{center}
\end{frame}

\begin{frame}[fragile]
   \frametitle{... And then create an instance of one: \\Triangle Wave}
   \begin{center}
   \begin{minipage}{100mm}
   \begin{lstlisting}
triangleWave = makeFSApprox(
   2,
   function(n){
      return(0);
   },
   function(n){
      if(n == 0){
         return(0);
      }
      return((8/((n^2)*(pi^2)))*sin(n*pi/2));
   },
   function(t){
      t <- t + 0.5;
      t <- t %% 2;
      if( ( floor(t) %% 2 ) == 0 ){
         return( 2*t - 1 );
       }
      return( -2*(t-1) + 1 );
   }
);
   \end{lstlisting}
   \end{minipage}
   \end{center}
\end{frame}

\begin{frame}[fragile]
   \frametitle{And one more instance: \\Square Wave}
   \begin{center}
   \begin{minipage}{100mm}
   \begin{lstlisting}
squareWave = makeFSApprox(
   2*pi,
   function(n){
      if(n == 0){
         return(1/2);
      }
      return((2/(n*pi))*sin((n*pi)/2));
   },
   function(n){
      return(0);
   },
   function(t){
      t0 <- abs(t) %% (2*pi);
      if( (t0 < pi/2) || (t0 > 3*pi/2) ){
         return(1);
      }
      else{
         return(0);
      }
   }
);
   \end{lstlisting}
   \end{minipage}
   \end{center}
\end{frame}

\begin{frame}[fragile]
   \frametitle{Define an interface for our class}
   \begin{center}
   \begin{minipage}{100mm}
   \begin{lstlisting}
getPlotParams <- function(obj, ...){
   UseMethod("getPlotParams");
}

Fn <- function(obj, ...){
   UseMethod("Fn");
}
   \end{lstlisting}
   \end{minipage}
   \end{center}
\end{frame}

\begin{frame}[fragile]
   \frametitle{Define implementations of our interface methods}
   \begin{center}
   \begin{minipage}{100mm}
   \begin{lstlisting}
getPlotParams.FSApprox <- function(fsa){
   return(list(
      xlab=expression(t),
      ylab=""
   ));
}

Fn.FSApprox <- function(fsa,N){
   return(
      function(t){
         return(
            summation(
               function(n) {
                  return(
                     (fsa$an(n) * sin( (2 * pi * n/fsa$T0 * t) + 
                        (90 * pi / 180))) +
                     (fsa$bn(n) * sin(  2 * pi * n/fsa$T0 * t  )) 
                  );
               }, 
               0,N
            )
         );
      }
  );
}
   \end{lstlisting}
   \end{minipage}
   \end{center}
\end{frame}

\begin{frame}[fragile]
   \frametitle{Define some helper methods we're going to need in just a minute}
   \begin{center}
   \begin{minipage}{100mm}
   \begin{lstlisting}
getSamplePoints <- function(t0, t1, frequency){
   #Nyquist says to use this many samples...
   n <- (2*frequency) * (t1 - t0);
   #But, Nyquist was reconstructing analog signals, not connecting 
   #dots with straight lines.
   n <- 50*n;
   return(linspace(t0,t1,n));
}

#NOTE: R does not optimize tail recursion
summation <- function(f, n, N){
   sum <- 0;
   for(i in n:N){
      sum <- sum + f(i);
   }
   return(sum);
}
   \end{lstlisting}
   \end{minipage}
   \end{center}
\end{frame}

\begin{frame}[fragile]
   \frametitle{Provide an implementation of a method we wish to override}
   \begin{center}
   \begin{minipage}{100mm}
   \begin{lstlisting}
plot.FSApprox <- function(fsa,t0,t1,n){
   t <- getSamplePoints(t0,t1,n/fsa$T0) ;
   plot( t, Fn(fsa,n)(t), 
         type='l', col="blue", 
         xlab=getPlotParams(fsa)$xlab, 
         ylab=getPlotParams(fsa)$ylab);
   lines( t, 
         sapply( t, fsa$trueFn),
         type='l', col="red");
   legend(  "topright", 
            legend=sapply(
               c(bquote(F[.(n)](t)), bquote(x(t))),
               as.expression
            ), 
            fill=c("blue", "red"),
            y.intersp=1.5,
            bg="white");
}
   \end{lstlisting}
   \end{minipage}
   \end{center}
\end{frame}

\begin{frame}[fragile]
   \frametitle{Now we can make the following calls to R's built in \emph{plot} function, which generate the plots on the next four slides}
   \begin{center}
   \begin{minipage}{100mm}
   \begin{lstlisting}
plot(squareWave,-2*pi,2*pi,10);
plot(squareWave,-2*pi,2*pi,50);
plot(triangleWave, -pi/2, pi/2, 3);
plot(triangleWave, -pi/2, pi/2, 20);
   \end{lstlisting}
   \end{minipage}
   \end{center}
\end{frame}

\begin{frame}[fragile]
   \frametitle{Approximation of Square Wave (10 Harmonics)}
   \begin{figure}
   \includegraphics[scale=0.40]{squareWave10.pdf}
   \end{figure}
\end{frame}

\begin{frame}[fragile]
   \frametitle{Approximation of Square Wave (50 Harmonics)}
   \begin{figure}
      \caption{Note: Ask me about the Gibbs phenomenon and uniform vs. pointwise convergence! (I'll answer if there's time.)}
      \includegraphics[scale=0.40]{squareWave50.pdf}
   \end{figure}
\end{frame}

\begin{frame}[fragile]
   \frametitle{Approximation of Triangle Wave (3 Harmonics)}
   \begin{figure}
   \includegraphics[scale=0.40]{triangleWave3.pdf}
   \end{figure}
\end{frame}

\begin{frame}[fragile]
   \frametitle{Approximation of Triangle Wave (20 Harmonics)}
   \begin{figure}
   \includegraphics[scale=0.40]{triangleWave20.pdf}
   \end{figure}
\end{frame}

\begin{frame}
   \frametitle{Questions?}
   \begin{center}
      {\fontsize{50}{60}\selectfont ???}
   \end{center}
\end{frame}

\begin{frame}
   \frametitle{Gibbs Phenomenon / Uniform vs. Pointwise Convergence}
   \begin{block}{}
      \begin{itemize}
         \item The overshoot seen at the discontinuity in our square wave will never die out (Gibbs). 
         \item This keeps our Square Wave approximation from being Uniformly Convergent (though it is Pointwise Converget).
         \item Our Triangle Wave approximation is Uniformly Convergent. Uniform Convergence is stronger than Pointwise Convergence.
      \end{itemize}
   \end{block}
\end{frame}
\end{document}
